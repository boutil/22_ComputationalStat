\documentclass[a4paper,10pt,fleqn]{article}

\usepackage{a4wide,amsmath,amsthm,amssymb,bbm,fancyhdr}
\usepackage{ifthen,color,enumerate,comment,dsfont,pdfsync,framed,todonotes,enumitem}

\newcommand{\titre}[1]{\textbf{\textsc{#1}}}

\RequirePackage[T1]{fontenc}

\usepackage[latin1]{inputenc}
\usepackage{graphicx}
\usepackage{dsfont}
\usepackage{enumitem}
\newcommand{\eqsp}{\,}
\newcommand{\R}{\ensuremath{\mathbb{R}}}
\newcommand{\calF}{\mathcal{F}}
\newcommand{\rmd}{\mathrm{d}}
\newcommand{\N}{\mathbb{N}}
\newcommand{\rset}{\ensuremath{\mathbb{R}}}
\renewcommand{\P}{\ensuremath{\operatorname{P}}}
\newcommand{\bP}{\mathbb{P}}
\newcommand{\E}{\ensuremath{\mathbb{E}}}
\newcommand{\rme}{\ensuremath{\mathrm{e}}}
\newcommand{\calH}{\ensuremath{\mathcal{H}}}
\newcommand{\xset}{\ensuremath{\mathsf{X}}}
\newcommand{\V}{\ensuremath{\mathbb{V}}}
\newcommand{\Sb}{\ensuremath{\mathbb{S}}}
\newcommand{\gaus}{\ensuremath{\mathcal{N}}}
\newcommand{\HH}{\ensuremath{\mathcal{H}}}
\newcommand{\F}{\ensuremath{\mathcal{F}}}
\newcommand{\W}{\ensuremath{\mathcal{W}}}
\newcommand{\X}{\ensuremath{\mathcal{X}}}
\newcommand{\1}{\ensuremath{\mathbbm{1}}}
\newcommand{\dlim}{\ensuremath{\stackrel{\mathcal{L}}{\longrightarrow}}}
\newcommand{\plim}{\ensuremath{\stackrel{\mathrm{P}}{\longrightarrow}}}
\newcommand{\PP}{\ensuremath{\mathbb{P}}}
\newcommand{\p}{\ensuremath{\mathbb{P}}}
\newcommand{\eps}{\varepsilon}
\newcommand{\bE}{\mathbb{E}}
\newcommand{\pa}[1]{\left(#1\right)}
\newcommand{\hatk}{\widehat K}
\newcommand{\f}{\varphi}
\newcommand{\Id}{\textsf{Id}}
\newcommand{\bfU}{\mathbf{U}}
\newcommand{\bfX}{\mathbf{X}}
\newcommand{\bfs}{\mathbf{\Sigma}}
\newcommand{\bfA}{\mathbf{A}}
\newcommand{\bfV}{\mathbf{V}}
\newcommand{\bfB}{\mathbf{B}}
\newcommand{\bfI}{\mathbf{I}}
\newcommand{\bfD}{\mathbf{D}}
\newcommand{\bfK}{\mathbf{K}}
\newcommand{\argmin}{\mathop{\textrm{argmin}}}
\newcommand{\argmax}{\mathop{\textrm{argmax}}}
\newcommand{\crit}{\mathop{\textrm{crit}}}
\newcommand{\C}{\mathcal{C}}
\newcommand{\pc}{\pi_{\mathcal{C}}}


% Style


\newtheorem{theorem}{Theorem}


\begin{document}

%\noindent Statistiques computationnelles \hfill Sorbonne Universit\'e% \\
% 2022-2023

\noindent\hrulefill

\begin{center}
\textsc{\'Echantillonneur de Gibbs}
\end{center}
\hrulefill

\medskip


\section*{Warm-up}
Soit $(X,Y)$ un couple de variables de loi gaussienne centr\'ee de matrice de covariance 
$$
\Sigma = \begin{pmatrix}
1 & \rho\\ \rho & 1
\end{pmatrix}\,,
$$
o\`u $\rho \in(0,1)$. 
\'Ecrire un \'echantillonneur de Gibbs permettant de simuler approximativement la loi de $(X,Y)$.

\section*{\'Echantillonneur pour un m\'elange gaussien}
Soit $K\geq 2$ et $n\geq 1$. On consid\`ere le vecteur al\'eatoire $(\theta,X,Z)$ o\`u $X = (X_1,\ldots,X_n)$ et $Z = (Z_1,\ldots,Z_n)$ ayant la loi suivante.
\begin{itemize}
\item On simule $p = (p_1,\ldots,p_K)$ un vecteur ayant la loi de densit\'e proportionnelle \`a (loi de Dirichlet) $p \mapsto \prod_{k=1}^Kp_k^{\gamma_k-1}$. %$\frac{\Gamma\left(\sum_{k=1}^K\gamma_k\right)}{\prod_{k=1}^K\Gamma(\gamma_k)}$
\item On simule $s^2_{1:K}$, mutuellement ind\'ependantes, et telles que pour tout $1\leq k \leq K$, $s^2_k$ a une loi inverse-gamma de param\`etres, $\lambda_k/2$ et $\beta_k/2$, i.e. de densit\'e proportionnelle \`a $u\mapsto u^{-\lambda_k/2-1}\exp(-\beta_k/(2u))$ sur $\mathbb{R}_+^*$.
\item Pour tout $1\leq k \leq K$, la loi conditionnelle de $m_k$ sachant $s_k^2$ est gaussienne de moyenne $\alpha_k$ et de variance $s_k^2/\lambda_k$.
\item Conditionnellement \`a $\theta = (p_{1},\ldots, p_K, m_{1}, \ldots, m_K, s^2_{1},\ldots, s^2_K)$, les $(Z_i,X_i)_{1\leq i \leq n}$ sont ind\'ependantes et telles que :
\begin{itemize}
\item pour tout $1\leq k \leq K$, $\mathbb{P}(Z_i = k|\theta) = p_k$ ;
\item conditionnellement \`a $\theta$ et $Z_i$, $X_i$ suit une loi gaussienne de moyenne $m_{Z_i}$ et de variance $s^2_{Z_i}$.
\end{itemize}
\end{itemize}
La densit\'e jointe peut alors s'\'ecrire :
$$
\pi: (\theta,x,z) \mapsto \pi(p) \left\{\prod_{k=1}^K\pi(s_k^2)\pi(m_k|s_k^2)\right\}\left\{\prod_{i=1}^n\pi(z_i|\theta)\pi(x_i|z_i,\theta)\right\}\,,
$$
o\`u $\pi(w_1|w_2)$ est une notation g\'en\'erique pour la densit\'e de la loi conditionnelle de la variable $W_1$ sachant $W_2$.
\begin{enumerate}
\item Montrer que la loi a posteriori de $\theta$ s'\'ecrit :
$$
\pi(\theta|x) \propto \pi(\theta)\prod_{i=1}^n\left(\sum_{k=1}^Kp_k \varphi_{m_k,s_k^2}(x_i)\right)\,,
$$
o\`u $\varphi_{m_k,s_k^2}$ est la densit\'e gaussienne de moyenne $m_k$ et de variance $s_k^2$.
\item \'Ecrire la densit\'e de la loi conditionnelle de $Z$ sachant $(X,\theta)$.
\item \'Ecrire la densit\'e de la loi conditionnelle de $\theta$ sachant $(Z,X)$.
\item \'Ecrire le pseudo-code de l'\'echantillonneur de Gibbs.
\end{enumerate}

%\section{Inf\'erence variationelle pour les mod\`eles exponentiels}
%On consid\`ere un couple de variables al\'eatoires $(Z,X)\in\mathbb{R}^d\times \mathbb{R}^m$. On note $(z,x) \mapsto p(z,x)$ la densit\'e jointe de ce couple par rapport \`a la mesure de Lebesgue. Nous souhaitons utiliser dans cet exercice une approche variationnelle pour estimer la loi a posteriori $p(z|x)$. Pour cela on se donne une famille de densit\'es sur $\mathbb{R}^d$:
%$$
%\mathcal{Q} = \left\{(z_1,\ldots,z_d)\mapsto \prod_{j=1}^dq_j(z_j)\eqsp;\eqsp q_j\; \mathrm{est\,une\,densit\acute e\,sur\,\mathbb{R}}\right\}\eqsp.
%$$
%\begin{enumerate}
%\item Rappeler l'algorithme CAVI (Coordinate Ascent Variational Inference) pour estimer it\'erativement $q^*$.
%\item Supposons que le mod\`ele soit tel que pour tout $j\in\mathbb{R}$, 
%$$
%p(z_j|z_{-j},x) = h(z_j)\mathrm{exp}(\eta(z_{-j})^Ts(z_j) - a(z_{-j}))\eqsp,
%$$ 
%o\`u $z_{-j} = (z_u)_{1\leqslant u\leqslant d, u \neq j}$ et o\`u $\eta$, $s$ et $a$ sont des fonctions connues (la d\'ependance en $x$ de ces fonctions est omise par simplicit\'e). Montrer que si les densit\'es $(q_u)_{1\leqslant u\leqslant d, u \neq j}$ sont fix\'ees alors la mise \`a jour de l'algorithme CAVI de la $j$-\`eme densit\'e est  (\`a une constante multiplicative pr\`es),
%$$
%q^*_j(z_j) \mapsto h(z_j) \mathrm{exp}\left\{\mathbb{E}_{-j}[\eta(Z_{-j})^Ts(Z_j)]\right\}\eqsp,
%$$
%o\`u $\mathbb{E}_{-j}$ est l'esp\'erance sous la loi de densit\'e $\prod_{u=1, u\neq j}^d q_u(z_u)$.
%\item La convergence de l'algorithme CAVI  d\'epend-elle de l'initialisation des densit\'e $(q_u)_{1\leqslant u\leqslant d}$ ?
%\end{enumerate}
\end{document}